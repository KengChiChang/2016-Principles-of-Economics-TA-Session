%!TEX TS-program = xelatex
%!TEX encoding = UTF-8 Unicode
\documentclass[12pt, xcolor=dvipsnames]{beamer}
\definecolor{slight}{gray}{0.9}
\fboxsep=10pt
\usecolortheme[named=Royal Blue]{structure}
\useinnertheme{circles}
\usepackage[no-math]{fontspec}
\usepackage{xltxtra, xunicode}
\usepackage[utf8]{inputenc}
%\usepackage[sc, osf]{mathpazo}
\usepackage[minionint, lf, mathtabular]{MinionPro}
\setmainfont[Mapping=tex-text]{Minion Web Pro}
\setsansfont[Mapping=tex-text]{Myriad Web Pro}
\setmonofont[Scale=MatchLowercase]{Source Code Pro}
\usefonttheme{professionalfonts}
%% 中文字配置
\usepackage[
CJKmath=true, indentfirst=false, PunctStyle={quanjiao},
CheckSingle=true, SlantFont, BoldFont
]{xeCJK}
\setCJKmainfont[Scale=0.9, BoldFont=Hiragino Mincho ProN W6]{Hiragino Mincho ProN W3}
%\setCJKmainfont[Scale=0.9, BoldFont=Noto Sans CJK JP Bold]{Noto Sans CJK JP Medium}
\setCJKsansfont[Scale=0.9, BoldFont=Hiragino Sans W6]{Hiragino Sans W4}
%\setCJKsansfont[Scale=0.9, BoldFont=Hiragino Sans CNS W6]{Hiragino Sans CNS W3}
%\setCJKsansfont[Scale=0.9, BoldFont=Hiragino Sans W7]{Hiragino Sans W4}
%\setCJKsansfont[Scale=0.9, BoldFont=Source Han Sans UI TC Bold]{Source Han Sans UI TC Regular}
%\setCJKsansfont[Scale=0.9, BoldFont=PingFang TC Semibold]{PingFang TC Regular}
\setCJKmonofont[Scale=0.9, BoldFont=Yuanti TC Regular]{Hiragino Maru Gothic ProN W4}
\usepackage{fancyvrb, attachfile2, pstricks}
\usepackage{graphicx}
\setbeamerfont{page number in head/foot}{size=\tiny}
\setbeamertemplate{footline}[frame number]
\usepackage{xmpmulti, booktabs, multicol}
\setbeamertemplate{navigation symbols}{}
\let\WriteBookmarks\relax
\usepackage{dcolumn}
\newcolumntype{.}[1]{D{.}{.}{#1}}
\newcolumntype{,}[1]{D{,}{,}{#1}}

\linespread{1.25}

\setbeamersize{text margin left=.8em, text margin right=.6em}

\makeatletter
\defbeamertemplate{itemize item}{mycircle}{\LARGE\raise-1.6pt\hbox{\textbullet}}
\makeatother

\setbeamertemplate{itemize item}[mycircle]
\setbeamertemplate{itemize subitem}[triangle]
\setlength\leftmargini{1.3em}
\setlength\leftmarginii{1em}


%\CTXFR
\title{\bf{\Huge {}\\[-2mm] Principles of Economics \\[2mm] Review Session}}
\author{{\Large 張耕齊\\[2mm] Keng-Chi Chang}}
\institute{{}\\[-7mm]\footnotesize\tt{<r03323070@ntu.edu.tw>}\\[2mm]}
\date{\large 2016.10.26}
\begin{document}
\fontsize{12}{14pt}\selectfont

\begin{frame}
\titlepage
\end{frame}

\begin{frame}
\frametitle{\bf §6.1--6.2 How Producers Make Choices}
\begin{itemize}
	\item Producers choose a quantity $q$ that maximizes profit $\pi=TR-TC$
	\item Marginal analysis tells us that $\pi$ is maximized when $MR=MC$
	\item In perfect competition, producers take price as given, so $MR=P$
	\item So producers produces at quantities such that $P=MC(q)$
	\item Hence firm's supply is basically $MC$, but under some conditions
\end{itemize}
\end{frame}

\begin{frame}
\frametitle{\bf §6.4--6.6 Short Run and Long Run}
\begin{itemize}
	\item Sunk cost: When a cost is irreversible, it is sunk 付出的愛收不回
	\item For a rational decision maker, sunk costs are irrelevant, since no matter what you do now, you will always have to pay that cost 
	\item In the short run, producers can always choose to shutdown
	\begin{itemize}
		\item Fixed costs are sunk in the short run, so shutdown if $P<AVC$
		\item So short run supply is the part of $MC$ that is above $AVC$
	\end{itemize}
	\item In the long run, producers can also choose to exit
	\begin{itemize}
		\item Since fixed costs are no longer sunk, firms choose to exit if $P<ATC$
		\item Hence long run supply is the part of $MC$ that is above $ATC$
		\item If we allow free entry, other firms may choose to enter if $P>ATC$
		\item Under free entry and exit, long run supply is the part of $MC$ that equals to $ATC$, which is $\min ATC$
	\end{itemize}
\end{itemize}
\end{frame}

\begin{frame}
\frametitle{\bf §7.1--7.5 Perfect Competition and Welfare}
\begin{itemize}
	\item Pareto improvement: A change of allocation that makes some people better off and others unchanged
	\item Pareto efficient: No Pareto improvement can be made, so any allocation change will make at least one person worse off
	\item First Welfare Theorem: Equilibrium outcome under perfect competition maximizes social surplus, which is Pareto efficient
	\begin{itemize}
		\item Adam Smith theorem, the power of the invisible hand
	\end{itemize}
	\item Distortions to a perfect competitive market usually decrease social welfare, we call them deadweight loss 無謂損失
	\item Pareto efficiency is silent on equity
	\begin{itemize}
		\item Unequal allocations may still be Pareto efficient
		\item Other than efficiency, we may also care about equity
	\end{itemize}
\end{itemize}
\end{frame}


\end{document}